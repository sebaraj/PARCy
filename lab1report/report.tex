\documentclass[12pt]{article}
\usepackage{graphicx}
\usepackage{amsfonts}
\usepackage{amsmath}
\usepackage{amssymb}
\usepackage{enumitem}
\usepackage{xcolor}

\usepackage[a4paper, margin=1in]{geometry}

\setlength{\headheight}{15pt}
\usepackage{fancyhdr}
\pagestyle{fancy}
\fancyhf{}
\fancyhead[L]{CPSC/EENG 420 - Lab Report 1}
\fancyhead[R]{Bryan SebaRaj  \thepage}

\linespread{1.5} 

\title{EENG 420 - Computer Architecture - Lab 1 Report}
\author{Bryan SebaRaj \\[0.5em] Professor Abhishek Bhattacharjee}
\date{January 31, 2025}

\begin{document}

\maketitle

%
% Abstract: introductory paragraph summarizing the lab
% • Design: describe your implementation, justifications for design decisions (if any), deviations from the
% prescribed datapath, discussion of any extensions
% • Testing Methodology: describe how you tested the modules and your overall testing strategy (any
% corner cases?)
% • Evaluation: report your simulation results and cycle count
%

\section*{Abstract}

Simple arithmetic operations are taken for granted in modern computing from the
perspective of the common software engineer. However, at the hardware level,
these operations are non-trivial to implement. This lab 


\section*{Design}

Starting with the multiplication module, the prescribed datapath was followed.
However, rather than using strucutual components such as multiplexers to 


\section*{Testing Methodology}


\section*{Evaluation}



\end{document}


