\documentclass[12pt]{article}
\usepackage{graphicx}
\usepackage{amsfonts}
\usepackage{amsmath}
\usepackage{amssymb}
\usepackage{enumitem}
\usepackage{xcolor}

\usepackage[a4paper, margin=1in]{geometry}

\setlength{\headheight}{15pt}
\usepackage{fancyhdr}
\pagestyle{fancy}
\fancyhf{}
\fancyhead[L]{CPSC/EENG 420 - Lab Report 1}
\fancyhead[R]{Bryan SebaRaj  \thepage}

\linespread{1.5} 

\title{EENG 420 - Computer Architecture - Lab 1 Report}
\author{Bryan SebaRaj \\[0.5em] Professor Abhishek Bhattacharjee}
\date{January 31, 2025}

\begin{document}

\maketitle

%
% Abstract: introductory paragraph summarizing the lab
% • Design: describe your implementation, justifications for design decisions (if any), deviations from the
% prescribed datapath, discussion of any extensions
% • Testing Methodology: describe how you tested the modules and your overall testing strategy (any
% corner cases?)
% • Evaluation: report your simulation results and cycle count
%

\section*{Abstract}

Simple arithmetic operations are taken for granted in modern computing from the
perspective of the common software engineer. However, at the hardware level,
these operations are non-trivial to implement. This lab focused on the
implementation of a iterative multiplication and division module; Verilog was
employed as the HDL to construct the multiplication and divison submodules,
which each took in two 32-bit inputs and yielded a 64-bit output to be parsed
as the product, quotient, or remainder. The submodules were integrated into a
compound module, and the entire system was tested using a testbench. The
results of the simulation were analyzed to determine the cycle count of the
system.

After confirming the correctness of the multiplication and division modules, and 


\section*{Design}

Starting with the multiplication module, the prescribed datapath was followed.
However, rather than relying solely on structural components, such as
multiplexers, to construct the data path, behavioral Verilog was employed to
partially define the flow of data.



\section*{Testing Methodology}


\section*{Evaluation}

The simulation results are as follows:
%
% 0xdeadbeef * 0x10000000 = ?
% 0xf5fe4fbc / 0x00004eb6 = ?
% 0x0a01b044 % 0xffffb14a = ?
% 0xf5fe4fbc /u 0x00004eb6 = ? (unsigned)
% 0x0a01b044 %u 0xffffb14a = ? (unsigned)
%
\begin{center}
    \begin{tabular}{|c | c | c | c| c|} 
 \hline
 Register A & Operation & Register B & Result & Number of Cycles \\
 \hline\hline
 0xdeadbeef & * & 0x1000000 & 0xfdeadbeef0000000 & 33 \\ 
 \hline
 0xf5fe4fbc & / &0x00004eb6 & 0xffffdf75 & 33 \\ 
 \hline
 0x0a01b044 & \% & 0xffffb14a & 0x00003372 & 33 \\ 
 \hline
 0xf5fe4fbc & /u & 0x00004eb6 & 0x00032012 & 33 \\ 
 \hline
 0x0a01b044 & \%u & 0xffffb14a & 0x0a01b044 & 33 \\ 
 \hline

\end{tabular}
\end{center}
See Appendix A for a screenshot of the simulation results.

\section*{Appendix A}

\includegraphics[scale=0.3]{/Users/bryansebaraj/Desktop/Screenshot 2025-01-28 at 10.06.00 PM.png}




\end{document}


